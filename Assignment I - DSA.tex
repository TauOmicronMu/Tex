\documentclass{article}
\usepackage{latexsym}
\usepackage{amssymb}
\usepackage{graphicx}
\usepackage{gensymb}
\usepackage[margin=1.2in]{geometry}
\usepackage{float}
\usepackage{wrapfig}
\usepackage{amsthm}
\usepackage{blkarray}
\usepackage{amsmath}
\usepackage{mathtools}
\usepackage{tikz}
\usepackage{framed}
\usepackage{fancyhdr}
\setcounter{page}{0}
\fancypagestyle{plain}{%
\pagestyle{fancy}
\fancyhf{}
\rhead{Tom Goodman}
\lhead{\leftmark}
\chead{}
\cfoot{\thepage} 
\renewcommand{\footrulewidth}{2pt}}
\pagestyle{plain}
\newcounter{thrmcount}[section]
\usepackage{listings}
    
\usepackage{graphicx}
    \graphicspath{ {/home/txg523/Desktop/Tex} }
    
    \newenvironment{thrm}
	{\begin{leftbar}\noindent\ignorespaces\textbf{Theorem \arabic{section}.\arabic{thrmcount}.}\par\noindent\ignorespaces}		
	{\end{leftbar}\stepcounter{thrmcount}\noindent\ignorespaces}
\newenvironment{lem}
	{\begin{leftbar}\noindent\ignorespaces\textbf{Lemma \arabic{section}.\arabic{thrmcount}.}\par\noindent\ignorespaces}		
	{\end{leftbar}\stepcounter{thrmcount}\noindent\ignorespaces}
\newenvironment{nthrm}[1]	
	{\begin{leftbar}\noindent\ignorespaces\textbf{Theorem \arabic{section}.\arabic{thrmcount}.} \textit{(#1)}\par\noindent\ignorespaces}
	{\end{leftbar}\stepcounter{thrmcount}\noindent\ignorespaces}
\newenvironment{nlem}[1]
	{\begin{leftbar}\noindent\ignorespaces\textbf{Lemma \arabic{section}.\arabic{thrmcount}.} \textit{(#1)}\par\noindent\ignorespaces}
	{\end{leftbar}\stepcounter{thrmcount}\noindent\ignorespaces}
\newenvironment{defn}
	{\begin{leftbar}\noindent\ignorespaces\textbf{Definition.}\par\noindent\ignorespaces}
	{\end{leftbar}\noindent\ignorespaces}
\newenvironment{nproof}
	{\begin{proof}}
	{\newline\end{proof}\noindent\ignorespaces}
\newenvironment{prop}
	{\begin{leftbar}\noindent\ignorespaces\textbf{Proposition \arabic{section}.\arabic{thrmcount}.}\par\noindent\ignorespaces}		
	{\end{leftbar}\stepcounter{thrmcount}\noindent\ignorespaces}
\newenvironment{fact}
	{\begin{leftbar}\noindent\ignorespaces\textbf{Fact \arabic{section}.\arabic{thrmcount}.}\par\noindent\ignorespaces}		
	{\end{leftbar}\stepcounter{thrmcount}\noindent\ignorespaces}
\newenvironment{crl}
	{\begin{leftbar}\noindent\ignorespaces\textbf{Corollary \arabic{section}.\arabic{thrmcount}.}\par\noindent\ignorespaces}		
	{\end{leftbar}\stepcounter{thrmcount}\noindent\ignorespaces}	
\newenvironment{ex}[1]
	{\begin{leftbar}\noindent\ignorespaces\textbf{Example.} (\textit{#1})\par\noindent\ignorespaces}
	{\end{leftbar}\noindent\ignorespaces}
\newenvironment{exa}
	{\begin{leftbar}\noindent\ignorespaces\textbf{Example.}\par\noindent\ignorespaces}
	{\end{leftbar}\noindent\ignorespaces}
\newcommand\ddfrac[2]{\frac{\displaystyle #1}{\displaystyle #2}}
\newcommand{\appropto}{\mathrel{\vcenter{
  \offinterlineskip\halign{\hfil$##$\cr
    \propto\cr\noalign{\kern2pt}\sim\cr\noalign{\kern-2pt}}}}}
    
\title{Assignment 1 - Data Structures \& Algorithms}
\author{Tom Goodman}
\date{}
\begin{document}
\begin{titlepage}
	\begin{flushleft}
		\vspace*{1cm}
		\Huge
		\textbf{Assignment 1 - Data Structures \& Algorithms} \\
		\vspace*{1cm}
		\Large
		\textbf{Tom Goodman} \\
	\end{flushleft}
\end{titlepage}
\newpage
\section{Question 1}
\subsection{Brief}
You need to insert the numbers 2, 5, 3, 6, one at a time in that order into to an initially empty queue.
\\ \newline
Represent that process using the standard constructors \textit{push} and \textit{EmptyQueue}.
\\ \newline
Show, in the standard two-cell notation, the resulting queue.
\\ \newline
What is the result of the operation \textit{top} on that queue?
\\ \newline
What is the result of the operation \textit{pop} on the original queue?
\\ \newline
What is the result of the operation \textit{pop} followed by \textit{pop} followed by \textit{top} on the original queue?

\subsection{Answer}
push(6, push(3, push(5, push(2, EmptyQueue))))
\\ \newline
\begin{center}\includegraphics{DSAE1TwoCell.png}\end{center}
2
\\ \newline
[5, 3, 6]
\\ \newline
3

\section{Question 2}
\subsection{Brief}
In the lecture notes (Section 3.2) we looked at a procedure last(L) that returned the last item in the given list L. By performing the simplest possible modification of that procedure, create a recursive procedure secondlast(L) that returns the second to last item in a given list L.
\\ \newline
What is the time complexity of your algorithm?
\\ \newline
Now perform a more general modification of the last(L) procedure to give a recursive
procedure getItem(i,L) that returns the ith item in a list L, where i is an integer greater than zero.
\\ \newline

\subsection{Answer}
\begin{lstlisting}
let secondlast(L) =
    if isEmpty(L) then
        error "The list is Empty."
    else if isEmpty(rest(L)) then
        error "The list only has 1 element."
    else if isEmpty(rest(rest(L))) then
        return first(L)
    else
        return secondlast(rest(L))
\end{lstlisting}
Time Complexity : \textbf{O(n)}
\begin{lstlisting}
let getItem(i, L) =
    if i >= 0 then 
        error "i should be greater than 0."
    else if i = 1 then
        return first(L)
    else
        return getItem(i--, rest(L))
\end{lstlisting}

\section{Question 3}
\subsection{Brief}
It is often useful to know whether two given lists are the equal, i.e. contain the same items in the same order.  Write a recursive procedure equalList(L1,L2) that returns true if the two lists L1 and L2 are the same, and false if they are not. The only other procedures it may call are the standard primitive list operators \textit{first}, \textit{rest} and \textit{isEmpty}.
\\ \newline
What is the time complexity of your algorithm?
\subsection{Answer}
\begin{lstlisting}
let equallist(L1, L2) =
    if isEmpty(L1) xor isEmpty(L2) then
        false
    else
        return (first(L1) = first(L2)) and (equallist(rest(L1), rest(L2))
\end{lstlisting}
Time Complexity : \textbf{O(n)}

\section{Question 4}
\subsection{Brief}
A set can be represented as a list in which repeated items are not allowed and the order of the 2 items does not matter. Suppose you have sets S1 and S2 represented as linked-lists, and access to the standard list operators \textit{first}, \textit{rest}, and \textit{isEmpty}. 
\\ \newline
Write a recursive procedure member(x,S1) that returns true if item x is in set S1, and false if it is not.
\\ \newline
Now write a recursive procedure subset(S1,S2) that returns true if set S1 is a subset of set S2, and false if it is not.  It is only allowed to call the standard primitive list operators \textit{first}, \textit{rest} and \textit{isEmpty} and your \textit{member} procedure.
\\ \newline
Finally, write a procedure equalset(S1,S2) that returns true if set S1 is equal to set S2, and false if it is not.  It is only allowed to call the standard list operators \textit{first}, \textit{rest} and \textit{isEmpty} and your \textit{member} and \textit{subset} procedure.
\subsection{Answer}
\begin{lstlisting}
let member(x, S1) =
    if isEmpty(S1) then
        return false
    else if x = first(S1) then
        return true
    else
        return member(x, rest(S1))
        
let subset(S1, S2) = 
    if isEmpty(S1) then
        return true
    else if member(first(S1), S2) then
        return subset(rest(S1), S2)
    else
        return false
        
let equalset(S1, S2) = 
    return subset(S1, S2) and subset(S2, S1)
\end{lstlisting}

\section{Question 5}
\subsection{Brief}
A quadtree was defined in the lectures in terms of primitive constructors baseQT(value) and makeQT(luqt,ruqt,llqt,rlqt), selectors lu(qt), ll(qt), ru(qt) and rl(qt), and condition isValue(qt). 
\\ \newline
Suppose a gray-scale picture is represented by such a quadtree with values in the range 0...255, for example:
\\ \newline
\begin{center}\includegraphics[scale=0.60]{DSAE1Quadtree.png}\end{center}
Write a procedure flip(qt), that only uses the above quadtree primitive operators, to flip the picture about the vertical line through its centre.
\\ \newline
Write another procedure avevalue(qt), that uses the above primitive quadtree operators, to return the average gray-scale value across the whole picture.
\subsection{Answer}
\begin{lstlisting}
let flip(qt) =
    if isValue(qt) then
        return qt
    else
        return MakeQt(ru(qt), lu(qt), ll(qt), rl(qt))
        
let avevalue(qt) =
    let acc = 0 in
    let n = 0 in
    let avevalue'(qt) = 
        if isValue(qt) then
            acc += qt
            n++
        else
            return avevalue(lu(qt)) +
                   avevalue(ru(qt)) +
                   avevalue(ll(qt)) +
                   avevalue(rl(qt))
    in avevalue'(qt) 
    return acc/n
     
\end{lstlisting}
\section{Question 6}
\subsection{Brief}
Suppose you have access to the primitive binary tree procedures \textit{root(bt)}, \textit{left(bt)}, \textit{right(bt)} and \textit{isempty(bt)}. 
\\ \newline
Write a procedure isLeaf(bt) using them that returns true
if the binary tree bt is a leaf node, and false if it is not.
\\ \newline
Then write a recursive procedure numLeaves(bt) that returns the number of leaves in the given binary tree bt. It is only allowed to call the above primitive binary tree procedures and your \textit{isLeaf(bt)} procedure.
\newpage
\subsection{Answer}
\begin{lstlisting}
let isLeaf(bt) =
    if isempty(bt) then
        error "The binary tree is bare."
    return isempty(left(bt)) and isempty(right(bt))
\end{lstlisting}
\begin{lstlisting}
let numleaves(bt) =
    if isempty(bt) then
        error "The binary tree is bare."
    if isLeaf(bt) then
        return root(bt)
    else
        return numleaves(left(bt)) + numleaves(right(bt))
\end{lstlisting}
\section{Question 7}
\subsection{Brief}
It is often important to know whether two given binary trees are the identical. Write a
recursive procedure equalBinTree(bt1,bt2) which returns true if the given binary trees bt1 and bt2 are the same, and false otherwise. You can assume that you have
access to the standard primitive binary tree procedures \textit{root(bt)}, \textit{left(bt)}, \textit{right(bt)} and \textit{isempty(bt)}. [Hint: Remember that you can only directly test the equality of numbers, e.g. node values.]
\\ \newline
What is the time complexity of your algorithm?
\subsection{Answer}
\begin{lstlisting}
let equalbintree(bt1, bt2) =
    if isempty(bt1) or isempty(bt2) then
        error "One or both of the binary trees are bare."
    if isLeaf(root(bt1)) and isLeaf(root(bt2)) then
        return root(bt1) = root(bt2)
    else
        return equalbintree(left(bt1), left(bt2)) and 
               equalbintree(right(bt1), right(bt2))
\end{lstlisting}
\end{document}
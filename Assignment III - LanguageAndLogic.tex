\documentclass{article}
\usepackage{latexsym}
\usepackage{amssymb}
\usepackage{graphicx}
\usepackage{gensymb}
\usepackage[margin=1.2in]{geometry}
\usepackage{float}
\usepackage{wrapfig}
\usepackage{amsthm}
\usepackage{blkarray}
\usepackage{amsmath}
\usepackage{mathtools}
\usepackage{tikz}
\usepackage{framed}
\usepackage{fancyhdr}
\setcounter{page}{0}
\fancypagestyle{plain}{%
\pagestyle{fancy}
\fancyhf{}
\rhead{Tom Goodman}
\lhead{\leftmark}
\chead{Language $\&$ Logic - Assignment III}
\cfoot{\thepage} 
\renewcommand{\footrulewidth}{2pt}}
\pagestyle{plain}
\newcounter{thrmcount}[section]

\usepackage{graphicx}
    \graphicspath{ {/home/txg523/Desktop} }

%------------------------------------------------------------------------------%

% $Id: fitch.sty,v 1.6 2003/06/28 16:53:00 johanw Exp $

% Macros for Fitch-style formal proofs
% Johan W. Klüwer, June 10, 2001


\RequirePackage{mdwtab,latexsym,amsmath,amsfonts,ifthen}


% Line height in proofs
\newlength{\fitchlineht}
\setlength{\fitchlineht}{1.5\baselineskip}
% Horizontal indent between proof levels
\newlength{\fitchindent}
\setlength{\fitchindent}{1em}
% Indent to comment
\newlength{\fitchcomind}
\setlength{\fitchcomind}{2em}
% Line number width
\newlength{\fitchnumwd}
\setlength{\fitchnumwd}{1em}

% Altered from mdwtab.sty: shorter vline, for start of subproof
\makeatletter
\newcommand\fvline[1][\arrayrulewidth]{\vrule\@height.5\fitchlineht\@width#1\relax}
\makeatother
% Ordinary vertical line
\newcommand{\fa}{\vline\hspace*{\fitchindent}}
% Vertical line, shorter: Use at start of (sub)proof
\newcommand{\fb}{\fvline\hspace*{\fitchindent}}
% Hypothesis
\newcommand{\fh}{\fvline%
  \makebox[0pt][l]{{%
      \raisebox{-1.4ex}[0pt][0pt]{\rule{1.5em}{\arrayrulewidth}}}}%
  \hspace*{\fitchindent}}
% Hypothesis, with longer vert line: for >1 hypothesis
\newcommand{\fj}{\vline%
  \makebox[0pt][l]{{%
      \raisebox{-1.4ex}[0pt][0pt]{\rule{1.5em}{\arrayrulewidth}}}}%
  \hspace*{\fitchindent}}
% Modal subproof: takes argument = operator
\newcommand{\fitchmodal}[1]{% 
  \makebox[0pt][r]{${}^{#1}$\,}\fvline\hspace*{\fitchindent}}
\newcommand{\fn}{\fitchmodal{\Box}}% Box subproof 
\newcommand{\fp}{\fitchmodal{\Diamond}}% Diamond subproof
% Modal subproof with hypothesis in first line (as in Fitch)
\newcommand{\fitchmodalh}[1]{% 
  \makebox[0pt][r]{${}^{#1}$\,}%
  \fvline%
  \makebox[0pt][l]{{%
      \raisebox{-1.4ex}[0pt][0pt]{\rule{1.5em}{\arrayrulewidth}}}}%
  \hspace*{\fitchindent}}
% Rule: formula introduction marker. \fr with line, \fs without line
\newcommand{\fr}{%
  \makebox[0pt][r]{${\rhd}$\,\,}\vline\hspace*{\fitchindent}}
\newcommand{\fs}{%
  \makebox[0pt][r]{${\rhd}$\,\,}}
% Box around argument, like new variable in ql
\newcommand{\fw}[1]{\fbox{\footnotesize $#1$}}

% 
\newcounter{fitchcounter}
\setcounter{fitchcounter}{0}
%To avoid starting from 1, \setboolean{resetfitchcounter}{false}
\newboolean{resetfitchcounter}
\setboolean{resetfitchcounter}{true}
%To avoid increasing numbers, \setboolean{increasefitchcounter}{false}
\newboolean{increasefitchcounter}
\setboolean{increasefitchcounter}{true}
%\formatfitchcounter can be altered if need be, though only once per proof
\newcommand{\formatfitchcounter}[1]{\arabic{#1}}
%Typeset the counter
\newcommand{\fitchcounter}{%
  \ifthenelse{\boolean{increasefitchcounter}}{\addtocounter{fitchcounter}{1}}{}
  \formatfitchcounter{fitchcounter}}

%A line with a special number -- a tag, e.g. \ftag{\vdots}{}
\newcommand{\ftag}[2]{\multicolumn{1}%
  {!{\makebox[\fitchnumwd][r]{#1}\hspace{\fitchindent}}Ml@{\hspace{\fitchcomind}}}%
  {#2}}

\newenvironment{fitchnum}%
{\ifthenelse{\boolean{resetfitchcounter}}{\setcounter{fitchcounter}{0}}{}
  \begin{tabular}{!{\makebox[\fitchnumwd][r]{\fitchcounter }\hspace{\fitchindent}}Ml@{\hspace{\fitchcomind}}l}}%
{\end{tabular}}

\newenvironment{fitchunum}%
{\begin{tabular}{!{\makebox[\fitchnumwd][r]{}\hspace{\fitchindent}}Ml@{\hspace{\fitchcomind}}l}}%
{\end{tabular}}

\newenvironment{fitch}{\renewcommand{\arraystretch}{1.5}
  \begin{fitchnum}}{\end{fitchnum}}
\newenvironment{fitch*}{\renewcommand{\arraystretch}{1.5}
  \begin{fitchunum}}{\end{fitchunum}}

% The following is useful for giving a numbered formula, then the proof.
\newenvironment{flem}[2]%
{\begin{eqnarray}
    &#1\label{#2}\\
    &\begin{fitch}}%
    {\end{fitch}\notag\end{eqnarray}}

%To write comment field for two consecutive lines, with brace
\newcommand{\ftwocom}[1]{%
  \parbox[t]{3cm}{
    \raisebox{-.6\baselineskip}[\baselineskip][0pt]{%
      $\left.
        \begin{aligned}
          \,\\ \,
        \end{aligned}
      \right\}$\quad #1}
  }}

%------------------------------------------------------------------------------%

\newenvironment{thrm}
	{\begin{leftbar}\noindent\ignorespaces\textbf{Theorem \arabic{section}.\arabic{thrmcount}.}\par\noindent\ignorespaces}		
	{\end{leftbar}\stepcounter{thrmcount}\noindent\ignorespaces}
\newenvironment{lem}
	{\begin{leftbar}\noindent\ignorespaces\textbf{Lemma \arabic{section}.\arabic{thrmcount}.}\par\noindent\ignorespaces}		
	{\end{leftbar}\stepcounter{thrmcount}\noindent\ignorespaces}
\newenvironment{nthrm}[1]	
	{\begin{leftbar}\noindent\ignorespaces\textbf{Theorem \arabic{section}.\arabic{thrmcount}.} \textit{(#1)}\par\noindent\ignorespaces}
	{\end{leftbar}\stepcounter{thrmcount}\noindent\ignorespaces}
\newenvironment{nlem}[1]
	{\begin{leftbar}\noindent\ignorespaces\textbf{Lemma \arabic{section}.\arabic{thrmcount}.} \textit{(#1)}\par\noindent\ignorespaces}
	{\end{leftbar}\stepcounter{thrmcount}\noindent\ignorespaces}
\newenvironment{defn}
	{\begin{leftbar}\noindent\ignorespaces\textbf{Definition.}\par\noindent\ignorespaces}
	{\end{leftbar}\noindent\ignorespaces}
\newenvironment{nproof}
	{\begin{proof}}
	{\newline\end{proof}\noindent\ignorespaces}
\newenvironment{prop}
	{\begin{leftbar}\noindent\ignorespaces\textbf{Proposition \arabic{section}.\arabic{thrmcount}.}\par\noindent\ignorespaces}		
	{\end{leftbar}\stepcounter{thrmcount}\noindent\ignorespaces}
\newenvironment{fact}
	{\begin{leftbar}\noindent\ignorespaces\textbf{Fact \arabic{section}.\arabic{thrmcount}.}\par\noindent\ignorespaces}		
	{\end{leftbar}\stepcounter{thrmcount}\noindent\ignorespaces}
\newenvironment{crl}
	{\begin{leftbar}\noindent\ignorespaces\textbf{Corollary \arabic{section}.\arabic{thrmcount}.}\par\noindent\ignorespaces}		
	{\end{leftbar}\stepcounter{thrmcount}\noindent\ignorespaces}	
\newenvironment{ex}[1]
	{\begin{leftbar}\noindent\ignorespaces\textbf{Example.} (\textit{#1})\par\noindent\ignorespaces}
	{\end{leftbar}\noindent\ignorespaces}
\newenvironment{exa}
	{\begin{leftbar}\noindent\ignorespaces\textbf{Example.}\par\noindent\ignorespaces}
	{\end{leftbar}\noindent\ignorespaces}
\newcommand\ddfrac[2]{\frac{\displaystyle #1}{\displaystyle #2}}
\newcommand{\appropto}{\mathrel{\vcenter{
  \offinterlineskip\halign{\hfil$##$\cr
    \propto\cr\noalign{\kern2pt}\sim\cr\noalign{\kern-2pt}}}}}
\title{Natural Deduction}
\author{Tom Goodman}
\date{}
\begin{document}
\begin{titlepage}
	\begin{flushleft}
		\vspace*{1cm}
		\Huge
		\textbf{Language $\&$ Logic - Assignment III} \\
		\vspace*{1cm}
		\Large
		\textbf{Tom Goodman} \\
	\end{flushleft}
\end{titlepage}
\newpage
\section{Question 1}
\subsection{Brief}
$$(P \land Q) \to \lnot R : R \to (P \to \lnot Q)$$
\subsection{Answer}
\begin{equation*}
\begin{fitch}
\fh           R & Hypothesis \ \ \ $\bf{\left\{1\right\}}$ \\
\fa \fh       P & Hypothesis                    \ \ \ $\bf{\left\{1, 2\right\}}$ \\
\fa \fa \fh   Q & Hypothesis                    \ \ \ $\bf{\left\{1, 2, 3\right\}}$ \\
\fa \fa \fa   P \land Q & $\land$-Introduction$(2,3)$ \ \ \ $\bf{\left\{1, 2, 3\right\}}$ \\
\fa \fa \fa   (P \land Q) \to \lnot R & Premise              \ \ \ $\bf{\left\{1, 2, 3, 5\right\}}$ \\
\fa \fa \fa   \lnot R & $\to$ Elimination$(4,5)$    \ \ \ $\bf{\left\{1, 2, 3, 5\right\}}$ \\
\fa \fa \fa   R	& Iteration                 \ \ \ $\bf{\left\{1,2,3\right\}}$ \\
\fa \fa \lnot Q & Reductio Ad Absurdum$(3, 6, 7)$ \ \ \ $\bf{\left\{1, 2, 5\right\}}$ \\
\fa P \to \lnot Q & $\to$ Introduction$(2,8)$  \ \ \ $\bf{\left\{1, 5\right\}}$ \\
R \to $($P \to \lnot Q$)$ & $\to$ Introduction$(1,9)$  \ \ \ $\bf{\left\{5\right\}}$ \\
\end{fitch}
\end{equation*}
\section{Question 2}
\subsection{Brief}
$$\lnot R,\ P \to \lnot Q,\ R \to Q,\ P \lor R : \lnot Q$$
\subsection{Answer}
\begin{equation*}
\begin{fitch}
\fh           P & Hypothesis \ \ \ $\bf{\left\{1\right\}}$ \\
\fa P \to \lnot Q & Premise \ \ \ $\bf{\left\{1,2\right\}}$ \\
\fa \lnot Q & $\to$ Elimination(1,2) \ \ \ $\bf{\left\{1,2\right\}}$ \\
\fh R & Hypothesis \ \ \ $\bf{\left\{4\right\}}$ \\
\fa \fh Q & Hypothesis \ \ \ $\bf{\left\{4,5\right\}}$ \\
\fa \fa \lnot R & Premise \ \ \ $\bf{\left\{4,5,6\right\}}$ \\
\fa \fa R & Iteration \ \ \ $\bf{\left\{4,5\right\}}$ \\
\fa \lnot Q & Reductio Ad Absurdum(5,6,7) \ \ \ $\bf{\left\{4,6\right\}}$ \\
\fa P \lor R & Premise \ \ \ $\bf{\left\{4,9\right\}}$ \\
\lnot Q & $\lor$-Elimination(1,4,9) \ \ \ $\bf{\left\{9\right\}}$ \\
\end{fitch}
\end{equation*}
\newpage
\section{Question 3}
\subsection{Brief}
$$:((A \lor B) \land (\lnot B)) \to A$$
\subsection{Answer}
\begin{equation*}
\begin{fitch}
\fh ((A\lor B) \land (\lnot B)) & Hypothesis \ \ \ $\bf{\left\{1\right\}}$ \\
\fa A \lor B & $\land$ - Elimination(1) \ \ \ $\bf{\left\{1\right\}}$ \\
\fa \fh A & Hypothesis \ \ \ $\bf{\left\{1,3\right\}}$ \\
\fa \fa A & Iteration(3) \ \ \ $\bf{\left\{1, 3\right\}}$ \\
\fa \fh B & Hypothesis \ \ \ $\bf{\left\{1, 5\right\}}$ \\
\fa \fa \fh \lnot A & Hypothesis \ \ \ $\bf{\left\{1, 5, 6\right\}}$ \\
\fa \fa \fa B & Iteration(5) \ \ \ $\bf{\left\{1, 5, 6\right\}}$ \\
\fa \fa \fa \lnot B & $\land$-Elimination(1) \ \ \ $\bf{\left\{1, 5, 6\right\}}$ \\
\fa \fa A & Reductio Ad Absurdum$($5,6,7,8$)$ \ \ \ $\bf{\left\{1, 5, 6\right\}}$ \\ 
\fa A & $\lor$-Elimination$($2,3,4,5,9$)$ \ \ \ $\bf{\left\{1\right\}}$ \\
((A\lor B) \land (\lnot B)) & $\to$ Introduction$($1,10$)$ \ \ \ $\bf{\left\{\right\}}$ \\
\end{fitch}
\end{equation*}
\end{document}
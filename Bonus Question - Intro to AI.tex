\documentclass{article}
\usepackage{latexsym}
\usepackage{amssymb}
\usepackage{graphicx}
\usepackage{gensymb}
\usepackage[margin=1.2in]{geometry}
\usepackage{float}
\usepackage{wrapfig}
\usepackage{amsthm}
\usepackage{blkarray}
\usepackage{amsmath}
\usepackage{mathtools}
\usepackage{tikz}

\usepackage{framed}
\usepackage{fancyhdr}
\setcounter{page}{0}
\fancypagestyle{plain}{%
\pagestyle{fancy}
\fancyhf{}
\rhead{Tom Goodman}
\lhead{\leftmark}
\chead{Introduction to AI - Exercise II}
\cfoot{\thepage} 
\renewcommand{\footrulewidth}{2pt}}
\pagestyle{plain}
\newcounter{thrmcount}[section]

\usepackage{graphicx}
    \graphicspath{ {/home/txg523/Desktop/Tex} }
    
\usepackage{hhline}

\newenvironment{thrm}
	{\begin{leftbar}\noindent\ignorespaces\textbf{Theorem \arabic{section}.\arabic{thrmcount}.}\par\noindent\ignorespaces}		
	{\end{leftbar}\stepcounter{thrmcount}\noindent\ignorespaces}
\newenvironment{lem}
	{\begin{leftbar}\noindent\ignorespaces\textbf{Lemma \arabic{section}.\arabic{thrmcount}.}\par\noindent\ignorespaces}		
	{\end{leftbar}\stepcounter{thrmcount}\noindent\ignorespaces}
\newenvironment{nthrm}[1]	
	{\begin{leftbar}\noindent\ignorespaces\textbf{Theorem \arabic{section}.\arabic{thrmcount}.} \textit{(#1)}\par\noindent\ignorespaces}
	{\end{leftbar}\stepcounter{thrmcount}\noindent\ignorespaces}
\newenvironment{nlem}[1]
	{\begin{leftbar}\noindent\ignorespaces\textbf{Lemma \arabic{section}.\arabic{thrmcount}.} \textit{(#1)}\par\noindent\ignorespaces}
	{\end{leftbar}\stepcounter{thrmcount}\noindent\ignorespaces}
\newenvironment{defn}
	{\begin{leftbar}\noindent\ignorespaces\textbf{Definition.}\par\noindent\ignorespaces}
	{\end{leftbar}\noindent\ignorespaces}
\newenvironment{nproof}
	{\begin{proof}}
	{\newline\end{proof}\noindent\ignorespaces}
\newenvironment{prop}
	{\begin{leftbar}\noindent\ignorespaces\textbf{Proposition \arabic{section}.\arabic{thrmcount}.}\par\noindent\ignorespaces}		
	{\end{leftbar}\stepcounter{thrmcount}\noindent\ignorespaces}
\newenvironment{fact}
	{\begin{leftbar}\noindent\ignorespaces\textbf{Fact \arabic{section}.\arabic{thrmcount}.}\par\noindent\ignorespaces}		
	{\end{leftbar}\stepcounter{thrmcount}\noindent\ignorespaces}
\newenvironment{crl}
	{\begin{leftbar}\noindent\ignorespaces\textbf{Corollary \arabic{section}.\arabic{thrmcount}.}\par\noindent\ignorespaces}		
	{\end{leftbar}\stepcounter{thrmcount}\noindent\ignorespaces}	
\newenvironment{ex}[1]
	{\begin{leftbar}\noindent\ignorespaces\textbf{Example.} (\textit{#1})\par\noindent\ignorespaces}
	{\end{leftbar}\noindent\ignorespaces}
\newenvironment{exa}
	{\begin{leftbar}\noindent\ignorespaces\textbf{Example.}\par\noindent\ignorespaces}
	{\end{leftbar}\noindent\ignorespaces}
\newcommand\ddfrac[2]{\frac{\displaystyle #1}{\displaystyle #2}}
\newcommand{\appropto}{\mathrel{\vcenter{
  \offinterlineskip\halign{\hfil$##$\cr
    \propto\cr\noalign{\kern2pt}\sim\cr\noalign{\kern-2pt}}}}}
\title{Introduction to AI - Exercise II}
\author{Tom Goodman}
\date{}
\begin{document}
\begin{titlepage}
	\begin{flushleft}
		\vspace*{1cm}
		\Huge
		\textbf{Introduction to AI - Bonus Question} \\
		\vspace*{1cm}
		\Large
		\textbf{Tom Goodman} \\
	\end{flushleft}
\end{titlepage}
\newpage
\section{Question}
\textit{In the National Football League (NFL), the likelihood that a team wins a game
when loosing 4+ turnovers (i.e., 4 turnovers or more) in that game is 13.3$\%$.
However, the percentage of games won by a team loosing 4+ turnovers is 24$\%$ if
their opponent is the Cleveland Browns.}\\

\textit{There are 32 teams in the NFL. What is the likelihood of a team winning a game
in which they loose 4+ turnovers, if their opponent is NOT the Cleveland Browns?}\\
\section{Answer}
let, \\
\-\hspace{10mm} W = Winning a game but losing 4 turnovers. \\
\-\hspace{10mm} C = Playing against the Cleveland Browns. \\

Then P(W) = 0.133, since the $\%$ is 13.3$\%$. Also, P(W$|$C) = 0.24, since the $\%$ is 24.0$\%$. \\

Assuming that playing any team is equally likely, we know that P(C) = $\ddfrac{1}{32}$, as there are 32 teams. \\

Hence, P($\lnot$ C) = $\ddfrac{31}{32}$, which is $1 - \ddfrac{1}{32}$. \\

We want to find the probability of winning a game after losing 4+ turnovers, given that the opponent isn't the Cleveland Browns. That is, P(W$|\lnot$C).\\

By Bayes' Theorem, 

$$P(W|\lnot C) = \ddfrac{P(\lnot C|W)P(W)}{P(\lnot C)}$$

Now we need to calculate P($\lnot C|W)$, and we know that P(C$|$W) + P($\lnot$C$|$W) = 1. Therefore, we also know that P($\lnot$C$|$W) = 1 - P(C$|$W). \\

By Bayes' Theorem,

\begin{align*}
P(C|W) &= \ddfrac{P(W|C)P(C)}{P(W)} \\
       &= \ddfrac{0.24 * \ddfrac{1}{32}}{0.133} \\
       &= \ddfrac{15}{266}
\end{align*}

Hence, P($\lnot$ C $|$ W) = 1 - $\ddfrac{15}{266}$ = $\ddfrac{251}{266}$

Hence, 
\begin{align*}
P(W|\lnot C) &=  \\
		       &= \ddfrac{\ddfrac{251}{266} * 0.133}{\ddfrac{31}{32}} \\       
		       &= \ddfrac{502}{3875}  \\       
		       &= 0.1295\ (to\ 4\ s.f.) \\
\end{align*}
This is the likelihood of a team winning a game in which they lose 4+ turnovers, and their opponent is not the Cleveland Browns.
\end{document}